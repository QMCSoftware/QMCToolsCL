\begin{itemize}
\item
\texttt{lat\_gen\_linear}: 
Lattice points in linear order
\begin{lstlisting}
	r (ints): replications
    n (ints): points
    d (ints): dimension
    g (array of ints): pointer to generating vector of size r*d
    x (array of floats): pointer to point storage of size r*n*d
\end{lstlisting}
\item
\texttt{lat\_gen\_gray}: 
Lattice points in Gray code order
\begin{lstlisting}
	r (ints): replications
    n (ints): points
    d (ints): dimension
    n_start (ints): starting index in sequence
    g (array of ints): pointer to generating vector of size r*d 
    x (array of floats): pointer to point storage of size r*n*d
\end{lstlisting}
\item
\texttt{lat\_gen\_natural}: 
Lattice points in natural order
\begin{lstlisting}
	r (ints): replications
    n (ints): points
    d (ints): dimension
    n_start (ints): starting index in sequence
    g (array of ints): pointer to generating vector of size r*d 
    x (array of floats): pointer to point storage of size r*n*d
\end{lstlisting}
\item
\texttt{lat\_shift\_mod\_1}: 
Shift mod 1 for lattice points
\begin{lstlisting}
	r (ints): replications
    n (ints): points
    d (ints): dimension
    r_x (ints): replications in x
    x (array of floats): lattice points of size r_x*n*d
    shifts (array of floats): shifts of size r*d
    xr (array of floats): pointer to point storage of size r*n*d
\end{lstlisting}
\item
\texttt{dnb2\_gmat\_lsb\_to\_msb}: 
Convert base 2 generating matrices with integers stored in Least Significant Bit order to Most Significant Bit order
\begin{lstlisting}
	r (ints): replications
    d (ints): dimension
    mmax (ints): columns in each generating matrix 
    tmaxes (array of ints): length r vector of bits in each integer of the resulting MSB generating matrices
    C_lsb (array of ints): original generating matrices of size r*d*mmax
    C_msb (array of ints): new generating matrices of size r*d*mmax
\end{lstlisting}
\item
\texttt{dnb2\_linear\_matrix\_scramble}: 
Linear matrix scrambling for base 2 generating matrices
\begin{lstlisting}
	r (ints): replications
    d (ints): dimension
    mmax (ints): columns in each generating matrix 
    r_C (ints): original generating matrices
    tmax_new (ints): bits in the integers of the resulting generating matrices
    S (array of ints): scrambling matrices of size r*d*tmax_new
    C (array of ints): original generating matrices of size r_C*d*mmax
    C_lms (array of ints): resulting generating matrices of size r*d*mmax
\end{lstlisting}
\item
\texttt{dnb2\_gen\_gray}: 
Binary representation of digital net in base 2 in Gray code order
\begin{lstlisting}
	r (ints): replications
    n (ints): points
    d (ints): dimension
    n_start (ints): starting index in sequence
    mmax (ints): columns in each generating matrix
    C (array of ints): generating matrices of size r*d*mmax
    xb (array of ints): binary digital net points of size r*n*d
\end{lstlisting}
\item
\texttt{dnb2\_gen\_natural}: 
Binary representation of digital net in base 2 in natural order
\begin{lstlisting}
	r (ints): replications
    n (ints): points
    d (ints): dimension
    n_start (ints): starting index in sequence
    mmax (ints): columns in each generating matrix
    C (array of ints): generating matrices of size r*d*mmax
    xb (array of ints): binary digital net points of size r*n*d
\end{lstlisting}
\item
\texttt{dnb2\_digital\_shift}: 
Digital shift base 2 digital net
\begin{lstlisting}
	r (ints): replications
    n (ints): points
    d (ints): dimension
    r_x (ints): replications of xb
    lshifts (array of ints): left shift applied to each element of xb
    xb (array of ints): binary base 2 digital net points of size r_x*n*d
    shiftsb (array of ints): digital shifts of size r*d
    xrb (array of ints): digital shifted digital net points of size r*n*d
\end{lstlisting}
\item
\texttt{dnb2\_integer\_to\_float}: 
Convert base 2 binary digital net points to floats
\begin{lstlisting}
	r (ints): replications
    n (ints): points
    d (ints): dimension
    tmaxes (array of ints): bits in integers of each generating matrix of size r
    xb (array of ints): binary digital net points of size r*n*d
    x (array of floats): float digital net points of size r*n*d
\end{lstlisting}
\item
\texttt{dnb2\_interlace}: 
Interlace generating matrices or transpose of point sets to attain higher order digital nets in base 2
\begin{lstlisting}
	r (ints): replications
    d_alpha (ints): dimension of resulting generating matrices 
    mmax (ints): columns of generating matrices
    d (ints): dimension of original generating matrices
    tmax (ints): bits in integers of original generating matrices
    tmax_alpha (ints): bits in integers of resulting generating matrices
    alpha (ints): interlacing factor
    C (array of ints): original generating matrices of size r*d*mmax
    C_alpha (array of ints): resulting interlaced generating matrices of size r*d_alpha*mmax
\end{lstlisting}
\item
\texttt{dnb2\_undo\_interlace}: 
Undo interlacing of generating matrices in base 2
\begin{lstlisting}
	r (ints): replications
    d (ints): dimension of resulting generating matrices 
    mmax (ints): columns in generating matrices
    d_alpha (ints): dimension of interlaced generating matrices
    tmax (ints): bits in integers of original generating matrices 
    tmax_alpha (ints): bits in integers of interlaced generating matrices
    alpha (ints): interlacing factor
    C_alpha (array of ints): interlaced generating matrices of size r*d_alpha*mmax
    C (array of ints): original generating matrices of size r*d*mmax
\end{lstlisting}
\item
\texttt{gdn\_linear\_matrix\_scramble}: 
Linear matrix scramble for generalized digital net
\begin{lstlisting}
	r (ints): replications 
    d (ints): dimension 
    mmax (ints): columns in each generating matrix
    r_C (ints): number of replications of C 
    r_b (ints): number of replications of bases
    tmax (ints): number of rows in each generating matrix 
    tmax_new (ints): new number of rows in each generating matrix 
    bases (array of ints): bases for each dimension of size r*d 
    S (array of ints): scramble matrices of size r*d*tmax_new*tmax
    C (array of ints): generating matrices of size r_C*d*mmax*tmax 
    C_lms (array of ints): new generating matrices of size r*d*mmax*tmax_new
\end{lstlisting}
\item
\texttt{gdn\_gen\_natural}: 
Generalized digital net where the base can be different for each dimension e.g. for the Halton sequence
\begin{lstlisting}
	r (ints): replications
    n (ints): points
    d (ints): dimension
    r_b (ints): number of replications of bases
    mmax (ints): columns in each generating matrix
    tmax (ints): rows of each generating matrix
    n_start (ints): starting index in sequence
    bases (array of ints): bases for each dimension of size r_b*d
    C (array of ints): generating matrices of size r*d*mmax*tmax
    xdig (array of ints): generalized digital net sequence of digits of size r*n*d*tmax
\end{lstlisting}
\item
\texttt{gdn\_gen\_natural\_same\_base}: 
Generalized digital net with the same base for each dimension e.g. a digital net in base greater than 2
\begin{lstlisting}
	r (ints): replications
    n (ints): points
    d (ints): dimension
    mmax (ints): columns in each generating matrix
    tmax (ints): rows of each generating matrix
    n_start (ints): starting index in sequence
    b (ints): common base
    C (array of ints): generating matrices of size r*d*mmax*tmax
    xdig (array of ints): generalized digital net sequence of digits of size r*n*d*tmax
\end{lstlisting}
\item
\texttt{gdn\_digital\_shift}: 
Digital shift a generalized digital net
\begin{lstlisting}
	r (ints): replications
    n (ints): points
    d (ints): dimension
    r_x (ints): replications of xdig
    r_b (ints): replications of bases
    tmax (ints): rows of each generating matrix
    tmax_new (ints): rows of each new generating matrix
    bases (array of ints): bases for each dimension of size r_b*d
    shifts (array of ints): digital shifts of size r*d*tmax_new
    xdig (array of ints): binary digital net points of size r_x*n*d*tmax
    xdig_new (array of ints): float digital net points of size r*n*d*tmax_new
\end{lstlisting}
\item
\texttt{gdn\_digital\_permutation}: 
Permutation of digits for a generalized digital net
\begin{lstlisting}
	r (ints): replications
    n (ints): points
    d (ints): dimension
    r_x (ints): replications of xdig
    r_b (ints): replications of bases
    tmax (ints): rows of each generating matrix
    tmax_new (ints): rows of each new generating matrix
    bmax (ints): common permutation size, typically the maximum basis
    perms (array of ints): permutations of size r*d*tmax_new*bmax
    xdig (array of ints): binary digital net points of size r_x*n*d*tmax
    xdig_new (array of ints): float digital net points of size r*n*d*tmax_new
\end{lstlisting}
\item
\texttt{gdn\_integer\_to\_float}: 
Convert digits of generalized digital net to floats
\begin{lstlisting}
	r (ints): replications
    n (ints): points
    d (ints): dimension
    r_b (ints): replications of bases 
    tmax (ints): rows of each generating matrix
    bases (array of ints): bases for each dimension of size r_b*d
    xdig (array of ints): binary digital net points of size r*n*d*tmax
    x (array of floats): float digital net points of size r*n*d
\end{lstlisting}
\item
\texttt{gdn\_interlace}: 
Interlace generating matrices or transpose of point sets to attain higher order digital nets
\begin{lstlisting}
	r (ints): replications
    d_alpha (ints): dimension of resulting generating matrices 
    mmax (ints): columns of generating matrices
    d (ints): dimension of original generating matrices
    tmax (ints): rows of original generating matrices
    tmax_alpha (ints): rows of interlaced generating matrices
    alpha (ints): interlacing factor
    C (array of ints): original generating matrices of size r*d*mmax*tmax
    C_alpha (array of ints): resulting interlaced generating matrices of size r*d_alpha*mmax*tmax_alpha
\end{lstlisting}
\item
\texttt{gdn\_undo\_interlace}: 
Undo interlacing of generating matrices
\begin{lstlisting}
	r (ints): replications
    d (ints): dimension of resulting generating matrices 
    mmax (ints): columns in generating matrices
    d_alpha (ints): dimension of interlaced generating matrices
    tmax (ints): rows of original generating matrices
    tmax_alpha (ints): rows of interlaced generating matrices
    alpha (ints): interlacing factor
    C_alpha (array of ints): interlaced generating matrices of size r*d_alpha*mmax*tmax_alpha
    C (array of ints): original generating matrices of size r*d*mmax*tmax
\end{lstlisting}
\item
\texttt{fwht\_1d\_radix2}: 
Fast Walsh-Hadamard Transform for real valued inputs.
FWHT is done in place along the last dimension where the size is required to be a power of 2.
Follows the divide-and-conquer algorithm described in https://en.wikipedia.org/wiki/Fast_Walsh%E2%80%93Hadamard_transform
\begin{lstlisting}
	d1 (ints): first dimension
    d2 (ints): second dimension
    n_half (ints): half of the last dimension along which FWHT is performed
    x (array of floats): array of size d1*d2*2n_half on which to perform FWHT in place
\end{lstlisting}
\item
\texttt{fft\_bro\_1d\_radix2}: 
Fast Fourier Transform for inputs in bit reversed order.
FFT is done in place along the last dimension where the size is required to be a power of 2.
Follows a decimation-in-time procedure described in https://www.cmlab.csie.ntu.edu.tw/cml/dsp/training/coding/transform/fft.html.
\begin{lstlisting}
	d1 (ints): first dimension
    d2 (ints): second dimension
    n_half (ints): half of the last dimension of size n = 2n_half along which FFT is performed
    twiddler (array of floats): size n vector used to store real twiddle factors
    twiddlei (array of floats): size n vector used to store imaginary twiddle factors 
    xr (array of floats): real array of size d1*d2*n on which to perform FFT in place
    xi (array of floats): imaginary array of size d1*d2*n on which to perform FFT in place
\end{lstlisting}
\item
\texttt{ifft\_bro\_1d\_radix2}: 
Inverse Fast Fourier Transform with outputs in bit reversed order.
FFT is done in place along the last dimension where the size is required to be a power of 2.
Follows a procedure described in https://www.expertsmind.com/learning/inverse-dft-using-the-fft-algorithm-assignment-help-7342873886.aspx.
\begin{lstlisting}
	d1 (ints): first dimension
    d2 (ints): second dimension
    n_half (ints): half of the last dimension of size n = 2n_half along which FFT is performed
    twiddler (array of floats): size n vector used to store real twiddle factors
    twiddlei (array of floats): size n vector used to store imaginary twiddle factors 
    xr (array of floats): real array of size d1*d2*n on which to perform FFT in place
    xi (array of floats): imaginary array of size d1*d2*n on which to perform FFT in place
\end{lstlisting}
\end{itemize}